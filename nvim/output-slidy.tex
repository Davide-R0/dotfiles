% Options for packages loaded elsewhere
\PassOptionsToPackage{unicode}{hyperref}
\PassOptionsToPackage{hyphens}{url}
\documentclass[
]{article}
\usepackage{xcolor}
\usepackage{amsmath,amssymb}
\setcounter{secnumdepth}{-\maxdimen} % remove section numbering
\usepackage{iftex}
\ifPDFTeX
  \usepackage[T1]{fontenc}
  \usepackage[utf8]{inputenc}
  \usepackage{textcomp} % provide euro and other symbols
\else % if luatex or xetex
  \usepackage{unicode-math} % this also loads fontspec
  \defaultfontfeatures{Scale=MatchLowercase}
  \defaultfontfeatures[\rmfamily]{Ligatures=TeX,Scale=1}
\fi
\usepackage{lmodern}
\ifPDFTeX\else
  % xetex/luatex font selection
\fi
% Use upquote if available, for straight quotes in verbatim environments
\IfFileExists{upquote.sty}{\usepackage{upquote}}{}
\IfFileExists{microtype.sty}{% use microtype if available
  \usepackage[]{microtype}
  \UseMicrotypeSet[protrusion]{basicmath} % disable protrusion for tt fonts
}{}
\makeatletter
\@ifundefined{KOMAClassName}{% if non-KOMA class
  \IfFileExists{parskip.sty}{%
    \usepackage{parskip}
  }{% else
    \setlength{\parindent}{0pt}
    \setlength{\parskip}{6pt plus 2pt minus 1pt}}
}{% if KOMA class
  \KOMAoptions{parskip=half}}
\makeatother
\usepackage{color}
\usepackage{fancyvrb}
\newcommand{\VerbBar}{|}
\newcommand{\VERB}{\Verb[commandchars=\\\{\}]}
\DefineVerbatimEnvironment{Highlighting}{Verbatim}{commandchars=\\\{\}}
% Add ',fontsize=\small' for more characters per line
\newenvironment{Shaded}{}{}
\newcommand{\AlertTok}[1]{\textcolor[rgb]{1.00,0.00,0.00}{\textbf{#1}}}
\newcommand{\AnnotationTok}[1]{\textcolor[rgb]{0.38,0.63,0.69}{\textbf{\textit{#1}}}}
\newcommand{\AttributeTok}[1]{\textcolor[rgb]{0.49,0.56,0.16}{#1}}
\newcommand{\BaseNTok}[1]{\textcolor[rgb]{0.25,0.63,0.44}{#1}}
\newcommand{\BuiltInTok}[1]{\textcolor[rgb]{0.00,0.50,0.00}{#1}}
\newcommand{\CharTok}[1]{\textcolor[rgb]{0.25,0.44,0.63}{#1}}
\newcommand{\CommentTok}[1]{\textcolor[rgb]{0.38,0.63,0.69}{\textit{#1}}}
\newcommand{\CommentVarTok}[1]{\textcolor[rgb]{0.38,0.63,0.69}{\textbf{\textit{#1}}}}
\newcommand{\ConstantTok}[1]{\textcolor[rgb]{0.53,0.00,0.00}{#1}}
\newcommand{\ControlFlowTok}[1]{\textcolor[rgb]{0.00,0.44,0.13}{\textbf{#1}}}
\newcommand{\DataTypeTok}[1]{\textcolor[rgb]{0.56,0.13,0.00}{#1}}
\newcommand{\DecValTok}[1]{\textcolor[rgb]{0.25,0.63,0.44}{#1}}
\newcommand{\DocumentationTok}[1]{\textcolor[rgb]{0.73,0.13,0.13}{\textit{#1}}}
\newcommand{\ErrorTok}[1]{\textcolor[rgb]{1.00,0.00,0.00}{\textbf{#1}}}
\newcommand{\ExtensionTok}[1]{#1}
\newcommand{\FloatTok}[1]{\textcolor[rgb]{0.25,0.63,0.44}{#1}}
\newcommand{\FunctionTok}[1]{\textcolor[rgb]{0.02,0.16,0.49}{#1}}
\newcommand{\ImportTok}[1]{\textcolor[rgb]{0.00,0.50,0.00}{\textbf{#1}}}
\newcommand{\InformationTok}[1]{\textcolor[rgb]{0.38,0.63,0.69}{\textbf{\textit{#1}}}}
\newcommand{\KeywordTok}[1]{\textcolor[rgb]{0.00,0.44,0.13}{\textbf{#1}}}
\newcommand{\NormalTok}[1]{#1}
\newcommand{\OperatorTok}[1]{\textcolor[rgb]{0.40,0.40,0.40}{#1}}
\newcommand{\OtherTok}[1]{\textcolor[rgb]{0.00,0.44,0.13}{#1}}
\newcommand{\PreprocessorTok}[1]{\textcolor[rgb]{0.74,0.48,0.00}{#1}}
\newcommand{\RegionMarkerTok}[1]{#1}
\newcommand{\SpecialCharTok}[1]{\textcolor[rgb]{0.25,0.44,0.63}{#1}}
\newcommand{\SpecialStringTok}[1]{\textcolor[rgb]{0.73,0.40,0.53}{#1}}
\newcommand{\StringTok}[1]{\textcolor[rgb]{0.25,0.44,0.63}{#1}}
\newcommand{\VariableTok}[1]{\textcolor[rgb]{0.10,0.09,0.49}{#1}}
\newcommand{\VerbatimStringTok}[1]{\textcolor[rgb]{0.25,0.44,0.63}{#1}}
\newcommand{\WarningTok}[1]{\textcolor[rgb]{0.38,0.63,0.69}{\textbf{\textit{#1}}}}
\usepackage{longtable,booktabs,array}
\usepackage{calc} % for calculating minipage widths
% Correct order of tables after \paragraph or \subparagraph
\usepackage{etoolbox}
\makeatletter
\patchcmd\longtable{\par}{\if@noskipsec\mbox{}\fi\par}{}{}
\makeatother
% Allow footnotes in longtable head/foot
\IfFileExists{footnotehyper.sty}{\usepackage{footnotehyper}}{\usepackage{footnote}}
\makesavenoteenv{longtable}
\usepackage{graphicx}
\makeatletter
\newsavebox\pandoc@box
\newcommand*\pandocbounded[1]{% scales image to fit in text height/width
  \sbox\pandoc@box{#1}%
  \Gscale@div\@tempa{\textheight}{\dimexpr\ht\pandoc@box+\dp\pandoc@box\relax}%
  \Gscale@div\@tempb{\linewidth}{\wd\pandoc@box}%
  \ifdim\@tempb\p@<\@tempa\p@\let\@tempa\@tempb\fi% select the smaller of both
  \ifdim\@tempa\p@<\p@\scalebox{\@tempa}{\usebox\pandoc@box}%
  \else\usebox{\pandoc@box}%
  \fi%
}
% Set default figure placement to htbp
\def\fps@figure{htbp}
\makeatother
\setlength{\emergencystretch}{3em} % prevent overfull lines
\providecommand{\tightlist}{%
  \setlength{\itemsep}{0pt}\setlength{\parskip}{0pt}}
\usepackage{bookmark}
\IfFileExists{xurl.sty}{\usepackage{xurl}}{} % add URL line breaks if available
\urlstyle{same}
\hypersetup{
  hidelinks,
  pdfcreator={LaTeX via pandoc}}

\author{}
\date{}

\begin{document}

\begin{itemize}
\tightlist
\item
  Colorscheme: \texttt{catppuccin-frappe}
\item
  markdown renderer: \texttt{render-markdown.nvim}
\item
  markdown preview: \texttt{preview-markdown.nvim}
\end{itemize}

\section{shdka j}\label{shdka-j}

\subsection{sd a}\label{sd-a}

\subsubsection{asd asd}\label{asd-asd}

\paragraph{asdasd}\label{asdasd}

\subparagraph{sajdhas}\label{sajdhas}

ajs hda

\begin{center}\rule{0.5\linewidth}{0.5pt}\end{center}

\begin{center}\rule{0.5\linewidth}{0.5pt}\end{center}

con l'implementazione di cmp ha completamento automatico sia delle
liste, dei callouts, e dei blocchi di codice

\subsection{Liste:}\label{liste}

\begin{itemize}
\tightlist
\item
  ad sahd

  \begin{itemize}
  \tightlist
  \item
    sjd a

    \begin{itemize}
    \tightlist
    \item
      askdlj

      \begin{itemize}
      \tightlist
      \item
        ajshd
      \end{itemize}
    \end{itemize}
  \end{itemize}
\item[$\square$]
  Task to do

  \begin{itemize}
  \tightlist
  \item
    {[}-{]} Task work in progress
  \item[$\boxtimes$]
    Task done
  \end{itemize}
\item
  {[}\textasciitilde{]} Custom one
\end{itemize}

\begin{quote}
dasjhd sjkshd sakjhd kjashdksja
dhshhhhhhhhhhhhhhhhhhhhhhhhhhhhhhhhhhhhhhhhhhhhhhhhhhhhhhhhhhhhhhhhhhhhhhhhhhhhhhhhhhhhhhhhhhhhhhhhhjhhhhhhhhhhkaj
shdkj askjdh ksjahd asdsadsad
\end{quote}

\paragraph{Codes}\label{codes}

\texttt{Inline\ code}

\begin{Shaded}
\begin{Highlighting}[]
\BuiltInTok{print}\NormalTok{(}\StringTok{"Hello !"}\NormalTok{)}
\end{Highlighting}
\end{Shaded}

\begin{Shaded}
\begin{Highlighting}[]
\KeywordTok{def}\NormalTok{ main() }\OperatorTok{{-}\textgreater{}} \VariableTok{None}\NormalTok{:}
    \BuiltInTok{print}\NormalTok{(}\StringTok{"Hello, World!"}\NormalTok{)}
\end{Highlighting}
\end{Shaded}

\begin{Shaded}
\begin{Highlighting}[]
\KeywordTok{fn}\NormalTok{ main() }\OperatorTok{\{}
    \PreprocessorTok{println!}\NormalTok{(}\StringTok{"Hello, World!"}\NormalTok{)}\OperatorTok{;}
\OperatorTok{\}}
\end{Highlighting}
\end{Shaded}

\begin{Shaded}
\begin{Highlighting}[]
\FunctionTok{ls} \AttributeTok{{-}ad}\NormalTok{ ./}
\FunctionTok{sudo}\NormalTok{ mount /dev/sda /mnt/myusb}
\end{Highlighting}
\end{Shaded}

\subsubsection{Callouts}\label{callouts}

This should be displayed even in markdown exporters

Importante: dopo il {[}!..{]} non lasciare spazi vuoti

\begin{quote}
{[}!NOTE{]} prova nota with \textgreater{}
\end{quote}

\begin{quote}
{[}!TIP{]} prova nota with \textgreater{}
\end{quote}

\begin{quote}
{[}!IMPORTANT{]} sd as
\end{quote}

\begin{quote}
{[}!WARNING{]} sadsd
\end{quote}

\begin{quote}
{[}!CAUTION{]} sd sa
\end{quote}

\begin{quote}
{[}!ABSTRACT{]} skdj as
\end{quote}

\begin{quote}
{[}!SUMMARY{]} asdh as
\end{quote}

\begin{quote}
{[}!TLDR{]} sad s
\end{quote}

\begin{quote}
{[}!INFO{]} sdj hsa
\end{quote}

\begin{quote}
{[}!TODO{]} sa djhas
\end{quote}

\begin{quote}
{[}!HINT{]} sada sd
\end{quote}

\begin{quote}
{[}!SUCCESS{]} sjadhas
\end{quote}

\begin{quote}
{[}!CHECK{]} jkh sad
\end{quote}

\begin{quote}
{[}!DONE{]} kj sdas
\end{quote}

\begin{quote}
{[}!QUESTION{]} hasdk j
\end{quote}

\begin{quote}
{[}!HELP{]} jshd s ad
\end{quote}

\begin{quote}
{[}!FAQ{]} jh sadh
\end{quote}

\begin{quote}
{[}!ATTENTION{]} sh sad
\end{quote}

\begin{quote}
{[}!FAILURE{]} jshdj h
\end{quote}

\begin{quote}
{[}!FAIL{]} jkash dasjh
\end{quote}

\begin{quote}
{[}!MISSING{]} jh asjhd
\end{quote}

\begin{quote}
{[}!DANGER{]} hg ksaas
\end{quote}

\begin{quote}
{[}!ERROR{]} jhsa kd
\end{quote}

\begin{quote}
{[}!BUG{]} jkh as d
\end{quote}

\begin{quote}
{[}!EXAMPLE{]} jh asd
\end{quote}

\begin{quote}
{[}!QUOTE{]} sadjsa d
\end{quote}

\begin{quote}
{[}!CITE{]} jh sakdjh
\end{quote}

\paragraph{Links}\label{links}


\begin{itemize}
\item
  \begin{figure}
  \centering
  \caption{Image}
  \end{figure}
\item
  \href{test.md}{Markdown File}
\item
  \href{test.py}{Python File}
\item
  \href{https://test.com}{Website}
\item
  {[}{[}wikilink{]}{]}
\item
  {[}{[}wikilink\textbar Wikilink Alias{]}{]}
\item
  {[}Reference{]}{[}example{]}
\item
  \href{mailto:user@test.com}{\nolinkurl{user@test.com}}
\end{itemize}

\subsection{Tabelle}\label{tabelle}

\begin{longtable}[]{@{}ll@{}}
\toprule\noalign{}
test! & prova \\
\midrule\noalign{}
\endhead
\bottomrule\noalign{}
\endlastfoot
ciao & coamskdj alkjsd \\
\end{longtable}

\subparagraph{Markdown preview}\label{markdown-preview}

\texttt{:MarkdownPreviewStart} @startuml Bob -\textgreater{} Alice :
hello @enduml

.

\begin{Shaded}
\begin{Highlighting}[]
\NormalTok{Bob {-}\textgreater{} Alice : hello}
\end{Highlighting}
\end{Shaded}

\begin{Shaded}
\begin{Highlighting}[]
\NormalTok{gantt}
\NormalTok{    dateFormat DD{-}MM{-}YYY}
\NormalTok{    axisFormat \%m/\%y}

\NormalTok{    title Example}
\NormalTok{    section example section}
\NormalTok{    activity :active, 01{-}02{-}2019, 03{-}08{-}2019}
\end{Highlighting}
\end{Shaded}

\begin{Shaded}
\begin{Highlighting}[]
\SpecialStringTok{$$ a\^{}3 }\SpecialCharTok{\textbackslash{}sqrt}\SpecialStringTok{\{as\} $$}
\end{Highlighting}
\end{Shaded}

\begin{quote}
provasdjh asjdh d jas s dksaj ldajsd s a dskd j sdhs
\end{quote}

\end{document}
